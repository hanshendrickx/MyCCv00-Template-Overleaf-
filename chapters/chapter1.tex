\chapter{Introduction}
\label{ch:introduction}

\section{Background}

This is the first chapter of your book. Begin by introducing the main topic and providing necessary background information.

\subsection{Motivation}

Explain why this topic is important and what motivated you to write this book. You can include citations like this \cite{example2023}.

\subsection{Objectives}

Clearly state the objectives and goals of your book:
\begin{itemize}
    \item Objective 1: To provide comprehensive coverage of the topic
    \item Objective 2: To present clear examples and explanations
    \item Objective 3: To guide readers through practical applications
\end{itemize}

\section{Structure of the Book}

Provide an overview of how the book is organized:

\begin{description}
    \item[Chapter 1] provides an introduction to the topic
    \item[Chapter 2] discusses the main concepts and theories
    \item[Chapter 3] presents practical applications and examples
\end{description}

\section{Mathematical Notation}

You can include mathematical equations in your book. For inline math, use \(E = mc^2\). For displayed equations:

\begin{equation}
    \int_{-\infty}^{\infty} e^{-x^2} dx = \sqrt{\pi}
\label{eq:gaussian}
\end{equation}

You can reference equations using \eqref{eq:gaussian}.

\section{Figures and Tables}

\begin{figure}[htbp]
    \centering
    % Uncomment when you have an image file:
    % \includegraphics[width=0.5\textwidth]{figures/example.png}
    \caption{An example figure placeholder. Replace with your actual figure.}
    \label{fig:example}
\end{figure}

Reference figures in text as shown in Figure~\ref{fig:example}.

\begin{table}[htbp]
    \centering
    \begin{tabular}{|l|c|r|}
        \hline
        \textbf{Column 1} & \textbf{Column 2} & \textbf{Column 3} \\
        \hline
        Row 1 & Data & 100 \\
        Row 2 & Data & 200 \\
        Row 3 & Data & 300 \\
        \hline
    \end{tabular}
    \caption{An example table}
    \label{tab:example}
\end{table}

\section{Theorems and Definitions}

\begin{definition}
A \textbf{set} is a collection of distinct objects.
\end{definition}

\begin{theorem}
This is an example theorem.
\end{theorem}

\begin{proof}
This is where you would provide the proof of the theorem.
\end{proof}

\section{Summary}

Summarize the key points introduced in this chapter and preview what comes next.
